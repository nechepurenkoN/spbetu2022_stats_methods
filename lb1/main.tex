\setcounter{page}{2}
\section*{Цели работы.}
Ознакомление с основными правилами формирования выборки и подготовки выборочных данных к статистическому анализу.

\section*{Постановка задачи.}
Осуществить формирование репрезентативной выборки заданного объема из имеющейся генеральной
совокупности экспериментальных данных. 
Осуществить последовательное преобразование полученной выборки в ранжированный, 
вариационный и интервальный ряды. 
Применительно к интервальному ряду построить и отобразить графически полигон, 
гистограмму и эмпирическую функцию распределения для абсолютных и относительных частот. 
Полученные результаты содержательно проинтерпретировать.

\section*{Порядок выполнения работы.}
\begin{enumerate}
    \item Выбрать программное обеспечение или язык программирования и обосновать его выбор.
    \item Выбрать двумерную генеральную совокупность, предварительно согласовав её с преподавателем. Указать, откуда была взята генеральная совокупность и предоставить ссылку.
    \item Из генеральной совокупности сформировать выборку заданного объёма в соответствии с полученным от преподавателя номером. Указать, каким образом была сформирована выборка и какого вида она получилась.
    \item Последовательно преобразовать выборку в ранжированный, вариационный и интервальный ряды. Результаты содержательно проинтерпретировать и сделать выводы.
    \item Для интервального ряда абсолютных частот построить и отобразить графически полигон, гистограмму и эмпирическую функцию. Сделать выводы.
    \item Аналогичные действия выполнить для интервального ряда относительных частот. Сравнить результаты и сделать выводы.
\end{enumerate}

\section*{Основные теоретические положения.}
Генеральная совокупность — это совокупность всех объектов или наблюдений, относительно которых исследователь намерен делать выводы при решении конкретной задачи. В ее состав включаются все объекты, которые подлежат изучению.

Выборка или выборочная совокупность — часть генеральной совокупности элементов, которая охватывается экспериментом.

Репрезентативность выборки описывает способность выборочных данных отражать структурные свойства совокупности, из которой они были извлечены. Т.е. даёт ответ на вопрос: можно ли в исследовании заменить совокупность на выборку без значимого ухудшения результатов анализа.
Выделяют качественную и количественную репрезентативность.
Для оценки репрезентативности выборки могут быть использованы как статистические,
так и не статистические методы.
