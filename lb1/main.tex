\setcounter{page}{2}
\section*{Цели работы.}
Ознакомление с основными правилами формирования выборки и подготовки выборочных данных к статистическому анализу.

\section*{Постановка задачи.}
Осуществить формирование репрезентативной выборки заданного объема из имеющейся генеральной
совокупности экспериментальных данных. 
Осуществить последовательное преобразование полученной выборки в ранжированный, 
вариационный и интервальный ряды. 
Применительно к интервальному ряду построить и отобразить графически полигон, 
гистограмму и эмпирическую функцию распределения для абсолютных и относительных частот. 
Полученные результаты содержательно проинтерпретировать.

\section*{Порядок выполнения работы.}
\begin{enumerate}
    \item Выбрать программное обеспечение или язык программирования и обосновать его выбор.
    \item Выбрать двумерную генеральную совокупность, предварительно согласовав её с преподавателем. Указать, откуда была взята генеральная совокупность и предоставить ссылку.
    \item Из генеральной совокупности сформировать выборку заданного объёма в соответствии с полученным от преподавателя номером. Указать, каким образом была сформирована выборка и какого вида она получилась.
    \item Последовательно преобразовать выборку в ранжированный, вариационный и интервальный ряды. Результаты содержательно проинтерпретировать и сделать выводы.
    \item Для интервального ряда абсолютных частот построить и отобразить графически полигон, гистограмму и эмпирическую функцию. Сделать выводы.
    \item Аналогичные действия выполнить для интервального ряда относительных частот. Сравнить результаты и сделать выводы.
\end{enumerate}

\section*{Основные теоретические положения.}
\textit{Генеральная совокупность} -- это совокупность всех объектов или наблюдений, относительно которых исследователь намерен делать выводы при решении конкретной задачи. В ее состав включаются все объекты, которые подлежат изучению.

\textit{Выборка} или \textit{выборочная совокупность} -- часть генеральной совокупности элементов, которая охватывается экспериментом.

\textit{Репрезентативность} выборки описывает способность выборочных данных отражать структурные свойства совокупности, из которой они были извлечены. Т.е. даёт ответ на вопрос: можно ли в исследовании заменить совокупность на выборку без значимого ухудшения результатов анализа.
Выделяют качественную и количественную репрезентативность.
Для оценки репрезентативности выборки могут быть использованы как статистические,
так и не статистические методы.

\textit{Вариационный ряд} -- это набор значений признака и их частот.
\textit{Ранжированный ряд} -- это упорядоченные (по возрастанию или убыванию) значения признака.
\textit{Интервальный ряд} -- это совокупность интервалов изучаемого признака, числа измерений, попадающих в выбранный интервал и/или
частот попадающих значений относительно объема выборки.

\textit{Полигон ряда частот} представляет собой линию, состоящую из точек $(x_i, f_i)$,
где $x_i$ -- значение середины $i$-того интервала, $f_i$ -- соответствующая частота.

\textit{Гистограмма частот} -- графическое представление распределения частот исследуемого
признака, образуемое соприкасающимися прямоугольниками, основаниями которых служат интервалы классов, а площади пропорциональны частотам этих классов.
Более формально, пусть $X_1, \dots, X_n$ - выборка, имеем разбиение $-\infty < a_0 < a_1 < \cdots < a{k-1} < a_k < \infty$,
\begin{gather*}
    n_i = \sum_{j=1}^n \mathds{1}_{\{X_j \in (a_{i-1}, a_i]\}}, i = 1, \dots, k
\end{gather*}
тогда
\begin{gather*}
    h(x) = \frac{n_i}{n\Delta a_i}, \Delta a_i = a_i - a_{i-1}, i = 1, \dots k
\end{gather*}
называется нормализованной гистограммой.

\textit{Эмпирической функцией распределения} называется функция равная:
\begin{gather*}
    F_n(x) = \frac{1}{n} \sum_{i=1}^n \mathds{1}_{\{x_i < x\}}
\end{gather*}
Эмпирическую функцию можно также строить по интервалам.

\section*{Выполнение работы.}
фыув