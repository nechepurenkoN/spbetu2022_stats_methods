\setcounter{page}{2}
\section*{Цели работы.}
Получение практических навыков вычисления интервальных статистических оценок параметров распределения выборочных данных и проверки «справедливости» статистических гипотез.

\section*{Постановка задачи.}
Для заданной надежности определить (на основании выборочных данных и результатов выполнения практической работы №2) границы доверительных интервалов для математического ожидания и среднеквадратичного отклонения случайной величины. Проверить гипотезу о нормальном распределении исследуемой случайной величины с помощью критерия Пирсона $\chi^2$.
Дать содержательную интерпретацию полученным результатам.

\section*{Порядок выполнения работы.}
\begin{enumerate}
    \item Вычислить точность и доверительный интервал для математического ожидания при неизвестном среднеквадратичном отклонении при заданном объёме выборки для доверительной точности $\gamma \in \{0.95, 0.99\}$.
    Сделать выводы.
    \item Для вычисления границ доверительного интервала для среднеквадратичного отклонения определить значение $q$при заданных $\gamma$ и $n$.
    Построить доверительные интервалы, сделать выводы.
    \item Проверить гипотезу о нормальности заданного распределения с помощью критерия $\chi^2$ (Пирсона). Для этого необходимо найти теоретические частоты и вычислить наблюдаемое значение критерия. Далее по заданному уровню значимости $\alpha = 0.05$ и числу степеней свободы найти критическую точку и сравнить с наблюдаемым значением. Сделать выводы.
\end{enumerate}

\section*{Основные теоретические положения.}
\textit{Интервальной оценкой} математического ожидания по выборочной
средней $\overline{x}_\text{B}$
при неизвестном среднеквадратическом отклонении $\sigma$ генеральной
совокупности служит доверительный интервал:
\begin{gather*}
    \overline{x}_\text{B} - \frac{t_\gamma s}{\sqrt{N}} < a < \overline{x}_\text{B} + \frac{t_\gamma s}{\sqrt{N}}
\end{gather*}

\textit{Интервальной оценкой} среднеквадратического отклонения $\sigma$ по
исправленной выборочной дисперсии служит доверительный интервал:
\begin{gather*}
    \frac{\sqrt{2N}}{\sqrt{2N-3} - t_\gamma} s < \sigma < \frac{\sqrt{2N}}{\sqrt{2N-3} - t_\gamma} s
\end{gather*}

\textit{Критерий Пирсона} – применяют для проверки гипотезы о соответствии
эмпирического распределения предполагаемому теоретическому
распределению $F(x)$.

Рассмотрим \textit{z-преобразование}
\begin{gather*}
    z = \frac{x - a}{\sigma}
\end{gather*}

Полученные величины используем в функции Лапласа
\begin{gather*}
    \Phi(z) = \frac{1}{2\pi} \int_0^z \exp(-\frac{t^2}{2})dt
\end{gather*}

С помощью рассмотренной функции можем вычислить теоретические частоты $p_i$:
\begin{gather*}
    p_i = \Phi(z_{i+1}) - \Phi(z_i) \\
    n_i' = p_i N
\end{gather*}

Затем можно вычислить наблюдаемое значение критерия и сравнить с табличным значением,
тем самым, принять или отклонить гипотезу о нормальном характере распределения.
\begin{gather*}
    \chi^2_\text{набл} = \sum_{i=1}^K \frac{(n_i - n_i')^2}{n_i'}
\end{gather*}

\section*{Выполнение работы.}
В предыдущей работы были получены следующие данные.
\noindent\textit{Таблица 1 -- Интервальный ряд роста мужчин}
\begin{longtable}{|p{4.5cm}|p{4.5cm}|p{3cm}|p{3cm}|}
    \hline
    Интервал        & Середина интервала & Абсолютная частота & Накопленная частота \\\hline
    [172.15 173.81) & 172.98             & 16                 & 16                  \\\hline
    [173.81 175.47) & 174.64             & 26                 & 42                  \\\hline
    [175.47 177.13) & 176.3              & 25                 & 67                  \\\hline
    [177.13 178.8)  & 177.96             & 15                 & 82                  \\\hline
    [178.8 180.46)  & 179.63             & 10                 & 92                  \\\hline
    [180.46 182.12) & 181.29             & 14                 & 106                 \\\hline
    [182.12 183.78] & 182.95             & 4                  & 110                 \\\hline
\end{longtable}

\noindent\textit{Таблица 2 -- Интервальный ряд роста женщин}
\begin{longtable}{|p{4.5cm}|p{4.5cm}|p{3cm}|p{3cm}|}
    \hline
    Интервал        & Середина интервала & Абсолютная частота & Накопленная частота \\\hline
    [158.29 160.01) & 159.15             & 8                  & 8                   \\\hline
    [160.01 161.74) & 160.88             & 24                 & 32                  \\\hline
    [161.74 163.46) & 162.6              & 26                 & 58                  \\\hline
    [163.46 165.19) & 164.32             & 16                 & 74                  \\\hline
    [165.19 166.91) & 166.05             & 19                 & 93                  \\\hline
    [166.91 168.64) & 167.77             & 11                 & 104                 \\\hline
    [168.64 170.36] & 169.5              & 6                  & 110                 \\\hline
\end{longtable}

Средний выборочный рост мужчин составил 176.83, женщин -- 163.73.
Исправленные СКО равны соответственно 2.86 и 2.74.

Зададим $\gamma = 0.95$, тогда табличное значение $t_\gamma$ равно $1.9819$.
По формулам выше найдем интервальные оценки для выборочных средних и СКО.
\begin{gather*}
    176.29 < a_\text{м} < 177.37 \\
    163.21 < a_\text{ж} < 164.25 \\
    2.54 < \sigma_\text{м} < 3.33 \\
    2.43 < \sigma_\text{ж} < 3.19
\end{gather*}

Проверим гипотезу о нормальном распределении исследуемых величин
с помощью критерия Пирсона $\chi^2$ (см. табл. 3 и 4).

\noindent\textit{Таблица 3 -- Теоретические частоты роста мужчин}
\begin{longtable}{|p{3cm}|p{3cm}|p{3cm}|p{3cm}|p{3cm}|}
    \hline
    $x_i$ & $x_{i+1}$ & $n_i$ & $p_i$ & $n_i'$ \\\hline
    $-\infty$ & 173.81 & 16 & 0.14 & 16.03 \\\hline
    173.81 & 175.47 & 26 & 0.17 & 18.89 \\\hline
    175.47 & 177.13 & 25 & 0.22 & 24.69 \\\hline
    177.13 & 178.8 & 15 & 0.21 & 23.26 \\\hline
    178.8 & 180.46 & 10 & 0.14 & 15.80 \\\hline
    180.46 & 182.12 & 14 & 0.07 & 7.73 \\\hline
    182.12 & $\infty$ & 4 & 0.03 & 3.56 \\\hline
\end{longtable}

\pagebreak

\noindent\textit{Таблица 4 -- Теоретические частоты роста женщин}
\begin{longtable}{|p{3cm}|p{3cm}|p{3cm}|p{3cm}|p{3cm}|}
    \hline
    $x_i$ & $x_{i+1}$ & $n_i$ & $p_i$ & $n_i'$ \\\hline
    $-\infty$ & 160.01 & 8 & 0.08 & 9.61 \\\hline
    160.01 & 161.74 & 24 & 0.14 & 16.04 \\\hline
    161.74 & 163.46 & 26 & 0.22 & 24.97 \\\hline
    163.46 & 165.19 & 16 & 0.24 & 26.53 \\\hline
    165.19 & 166.91 & 19 & 0.17 & 19.23 \\\hline
    166.91 & 168.64 & 11 & 0.08 & 9.51 \\\hline
    168.64 & $\infty$ & 6 & 0.03 & 4.07 \\\hline
\end{longtable}

В качестве критического значения будем использовать $\chi^2_\text{крит}=9.5$ при
уровне значимости $\alpha = 0.05$ и 4 степенях свободы.

Для мужского роста значение критерия равно 12.86, для женского -- 9.59.
Исходя из этого, гипотеза о нормальном характере распределения отклоняется для обоих величин $\chi^2_\text{набл} > \chi^2_\text{крит}$.

На самом деле, подобный результат совершенно ожидаем, ведь в построении выборке участвовали
110 <<самых высоких>> стран, таким образом был потерян <<левый хвост>>.

\section*{Выводы.}
В результате выполнения работы были получены границы интервальных оценок
для среднего роста мужчин и женщин, а также границы интервалов выборочного СКО.

Также была проверена гипотеза о нормальном характере распределения
рассматриваемых величин.
Значение критерия Пирсона оказалось выше критического значения, поэтому
гипотеза была отклонена для обеих величин.