\setcounter{page}{2}
\section*{Цели работы.}
Получение практических навыков нахождения точечных статистических оценок параметров распределения.

\section*{Постановка задачи.}
Для заданных выборочных данных вычислить с использованием метода моментов и условных вариант
точечные статистические оценки математического ожидания, дисперсии, среднеквадратичного
отклонения, асимметрии и эксцесса исследуемой случайной величины.
Полученные результаты содержательно проинтерпретировать.

\section*{Порядок выполнения работы.}
\begin{enumerate}
    \item Для интервального ряда, полученного в лабораторной работе №1 найти середины интервалов, а также накопленные частоты. Результаты занести в таблицу.
    \item Для полученных вариант вычислить условные варианты. Результаты занести в таблицу.
    \item Вычислить условные эмпирические моменты $M_i^*$ через условные варианты. С помощью условных эмпирических моментов вычислить центральные эмпирические моменты $\mu_i^*$. Полученные результаты занести в таблицу.
    \item Вычислить выборочные среднее и дисперсию с помощью стандартной формулы и с помощью условных вариант. Убедиться, что результаты совпадают.
    \item Найти статистическую оценку коэффициентов асимметрии и эксцесса. Сделать выводы.
    \item Дополнительное необязательное задание: для интервального ряда вычислить моду и медиану заданного распределения. Сделать выводы.
\end{enumerate}

\section*{Основные теоретические положения.}
\textit{Статистической оценкой} $\Theta^*$ неизвестного параметра теоретического
распределения $\Theta$ называется функция от наблюдаемых значений случайной величины:
\begin{gather*}
    \Theta^* = f(x_1, x_2, \dots, x_n)
\end{gather*}

Статистические оценки, определяемые одним числом, называются \textit{точечными}.

Оценка называется \textit{несмещенной}, если ее математическое ожидание равно
оцениваемому параметру $\Theta$ при любом объеме выборки $n$.

Начальным эмпирическим моментом $k$-того порядка называется среднее значение
$k$-х степеней элементов вариационного или интервального ряда:
\begin{gather*}
    \overline{M}_k = \frac{1}{N} \sum n_j x_j^k
\end{gather*}

Центральным эмпирическим моментом $k$-того порядка называется среднее значение
$k$-х степеней разностей $x_j - \overline{x}$ для вариационного или интервального
ряда.
\begin{gather*}
    \overline{m}_k = \frac{1}{N} \sum n_j (x_j - \overline{x})^k
\end{gather*}

Для упрощения вычислений используют условные моменты $k$-того порядка:
\begin{gather*}
    \overline{M}_k^* = \frac{1}{N} \sum n_j (\frac{x_j - C}{h})^k
    = \frac{1}{N} \sum n_j u_j^k
\end{gather*}
где $C$ -- середина интервала, принятого за условный ноль, $h$ -- длина интервала.

Центральные эмпирические моменты связаны с условными следующими соотношениями:
\begin{gather*}
    \overline{x} = \overline{M}_1 = \overline{M}_1^* + C \\
    \overline{m}_2 = (\overline{M}_2^* - (\overline{M}_1^*)^2) h^2 \\
    \overline{m}_3 = (\overline{M}_3^* - 3\overline{M}_2^* \overline{M}_1^*
    + 2(\overline{M}_1^*)^3)h^3 \\
    \overline{m}_4 = (\overline{M}_4^* - 4 \overline{M}_3^*\overline{M}_1^*
    + 6\overline{M}_2^* (\overline{M}_1^*)^2 - 3(\overline{M}_1^*)^4)h^4
\end{gather*}

Статистические оценки асимметрии и эксцесса вычисляются по формулам:
\begin{gather*}
    \overline{A}_s = \frac{\overline{m}_3}{s^3} \\
    \overline{E}_s = \frac{\overline{m}_3}{s^3} - 3
\end{gather*}
где $s^2$ -- несмещенная оценка дисперсии.

\section*{Выполнение работы.}
Для выполнения работы был выбран язык Python3 и среда Jupyter Notebook с сервисом
Google Colab.

В лабораторной работе №1 были получены интервальные ряды для мужского и женского роста.
Вычислим середины интервалов, а также накопленные частоты (см. табл. 1 и 2).

\noindent\textit{Таблица 1 -- Интервальный ряд роста мужчин}
\begin{longtable}{|p{4.5cm}|p{4.5cm}|p{3cm}|p{3cm}|}
    \hline
    Интервал        & Середина интервала & Абсолютная частота & Накопленная частота \\\hline
    [172.15 173.81) & 172.98             & 16                 & 16                  \\\hline
    [173.81 175.47) & 174.64             & 26                 & 42                  \\\hline
    [175.47 177.13) & 176.3              & 25                 & 67                  \\\hline
    [177.13 178.8)  & 177.96             & 15                 & 82                  \\\hline
    [178.8 180.46)  & 179.63             & 10                 & 92                  \\\hline
    [180.46 182.12) & 181.29             & 14                 & 106                 \\\hline
    [182.12 183.78] & 182.95             & 4                  & 110                 \\\hline
\end{longtable}

\noindent\textit{Таблица 2 -- Интервальный ряд роста женщин}
\begin{longtable}{|p{4.5cm}|p{4.5cm}|p{3cm}|p{3cm}|}
    \hline
    Интервал        & Середина интервала & Абсолютная частота & Накопленная частота \\\hline
    [158.29 160.01) & 159.15             & 8                  & 8                   \\\hline
    [160.01 161.74) & 160.88             & 24                 & 32                  \\\hline
    [161.74 163.46) & 162.6              & 26                 & 58                  \\\hline
    [163.46 165.19) & 164.32             & 16                 & 74                  \\\hline
    [165.19 166.91) & 166.05             & 19                 & 93                  \\\hline
    [166.91 168.64) & 167.77             & 11                 & 104                 \\\hline
    [168.64 170.36] & 169.5              & 6                  & 110                 \\\hline
\end{longtable}

Для вычисления условных вариант в качестве условного нуля возьмем середину 4 интервала как для мужчин,
так и для женщин.

Построим таблицы вычисления условных вариант согласно рассмотренной в лекции, только будем
использовать абсолютные частоты.
Результаты приведены в таблицах 3 и 4.

\noindent\textit{Таблица 3 -- Условные варианты роста мужчин}
\begin{longtable}{|p{1.7cm}|p{1.7cm}|p{1.7cm}|p{1.7cm}|p{1.7cm}|p{1.7cm}|p{1.7cm}|p{1.9cm}|}
    \hline
    $\overline{x}_i$ & $u$ & $n$ & $nu$ & $nu^2$ & $nu^3$ & $nu^4$ & $n(u+1)^4$ \\\hline
    172.98           & -3  & 16  & -48  & 144    & -432   & 1296   & 256        \\\hline
    174.64           & -2  & 26  & -52  & 104    & -208   & 416    & 26         \\\hline
    176.30           & -1  & 25  & -25  & 25     & -25    & 25     & 0          \\\hline
    177.96           & 0   & 15  & 0    & 0      & 0      & 0      & 15         \\\hline
    179.62           & 1   & 10  & 10   & 10     & 10     & 10     & 160        \\\hline
    181.28           & 2   & 14  & 28   & 56     & 112    & 224    & 1134       \\\hline
    182.94           & 3   & 4   & 12   & 36     & 108    & 324    & 1024       \\\hline
    \multicolumn{2}{|c|}{$\sum$} & 110 & -75 & 375 & -435 & 2295 & 2615 \\\hline
    \multicolumn{3}{|c|}{$M_k^*$} & -0.68 & 3.41 & -3.95 & 20.86 & - \\\hline
\end{longtable}

Выполним проверку по последнему столбцу: $2295 - 1740 + 2250 - 300 + 110 = 2615$ -- верно.

\noindent\textit{Таблица 4 -- Условные варианты роста женщин}
\begin{longtable}{|p{1.7cm}|p{1.7cm}|p{1.7cm}|p{1.7cm}|p{1.7cm}|p{1.7cm}|p{1.7cm}|p{1.9cm}|}
    \hline
    $\overline{x}_i$ & $u$ & $n$ & $nu$ & $nu^2$ & $nu^3$ & $nu^4$ & $n(u+1)^4$ \\\hline
    159.15           & -3  & 8   & -24  & 72     & -216   & 648    & 128        \\\hline
    160.87           & -2  & 24  & -48  & 96     & -192   & 384    & 24         \\\hline
    162.60           & -1  & 26  & -26  & 26     & -26    & 26     & 0          \\\hline
    164.32           & 0   & 16  & 0    & 0      & 0      & 0      & 16         \\\hline
    166.04           & 1   & 19  & 19   & 19     & 19     & 19     & 304        \\\hline
    167.77           & 2   & 11  & 22   & 44     & 88     & 176    & 891        \\\hline
    169.49           & 3   & 6   & 18   & 54     & 162    & 486    & 1536       \\\hline
    \multicolumn{2}{|c|}{$\sum$} & 110 & -39 & 311 & -165 & 1739 & 2899 \\\hline
    \multicolumn{3}{|c|}{$M_k^*$} & -0.35 & 2.83 & -1.5 & 15.81 & \\\hline
\end{longtable}

Выполним проверку по последнему столбцу: $1739 - 660 + 1866 - 156 + 110 = 2899$ -- верно.

Обозначим эмпирические моменты как $\mu_k$ (в лекциях и формулах в разделе основные теоретические положения
эмпирические моменты обозначались как $\overline{m}_k$).
Вычислим данные величины по формулам выше, получаем:
\begin{gather*}
    \mu_1^{\text{м}} = 176.83 \\
    \mu_2^{\text{м}} = 8.12 \\
    \mu_3^{\text{м}} = 10.93 \\
    \mu_4^{\text{м}} = 144.30 \\
    \mu_1^{\text{ж}} = 163.73 \\
    \mu_2^{\text{ж}} = 7.45 \\
    \mu_3^{\text{ж}} = 6.50 \\
    \mu_4^{\text{ж}} = 120.13
\end{gather*}

Вычислим среднее и дисперсию интервальных рядов с помощью стандартных формул:
\begin{gather*}
    \overline{x}_\text{в} = \frac{1}{n} \sum n_i \overline{x}_i \\
    D = \frac{1}{n} \sum n_i * (\overline{x}_i - \overline{x}_\text{в})^2
\end{gather*}

Результаты совпали.

Вычислим несмещенную оценку дисперсии по формуле:
\begin{gather*}
    s^2 = \frac{n}{n-1} D
\end{gather*}

Получаем следующие значения:
\begin{gather*}
    s^2_\text{м} = 8.20 \\
    s^2_\text{ж} = 7.52
\end{gather*}

По приведенным в разделе Основные теоретические положения формулам вычислим оценку коэффициентов асимметрии и эксцесса:
\begin{gather*}
    \overline{A}_\text{м} = 0.46 \\
    \overline{E}_\text{м} = 2.14 \\
    \overline{A}_\text{ж} = 0.31 \\
    \overline{E}_\text{ж} = 2.12
\end{gather*}

Выборки роста мужчин и женщин имеют правостороннюю асимметрию, что вполне ожидаемо, так как
были отобраны 110 стран с наиболее высоким населением в среднем.
Так как значение коэффициента эксцесса положительно, кривые распределений островершинные.

\section*{Выводы.}
В результате выполнения работы были найдены точечные статистические оценки параметров
распределения для заданной выборки.
С помощью метода условных эмпирических моментов были вычислены центральные эмпирические моменты
1-4 порядка.
Результаты были сравнены с результатами стандартных формул для 1 и 2 порядка.
Также были получены оценки коэффициентов асимметрии и эксцесса.